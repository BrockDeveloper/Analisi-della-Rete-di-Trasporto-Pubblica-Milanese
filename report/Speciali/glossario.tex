
\chapter*{Glossario}

% Il capitolo non numerato va aggiunto a mano alla table of contents (toc)
\addcontentsline{toc}{chapter}{Glossario}

\textbf{PTN} Dall'inglese \textit{Public Transport Network}, rete di trasporto pubblico.

\textbf{Rete Complessa} Consiste in una collezione di componenti, o nodi, che interagiscono tra loro. Queste reti sono utilizzate per modellare e analizzare una serie di sistemi complessi come le reti sociali, gli ecosistemi e le reti di trasporto pubblico.

\textbf{L-Space} Rappresentazione di una \textit{PTN} in cui ogni stazione viene rappresentata da un nodo e due stazioni servite successivamente da almeno una linea sono connesse da un arco. In questa rappresentazione il grafo risultate è simile alla struttura topologica della \textit{PTN}.

\textbf{Componente connessa} Sottoinsieme di nodi del grafo all’interno del quale esiste un cammino tra ogni coppia di nodi. La “giant component” è la più grande di queste.

\textbf{GCC} dall'inglese \textit{Giant Connected Component}, è la più grande componente connessa del grafo, ossia il gruppo di nodi ancora collegati tra loro dopo una serie di rimozioni o attacchi.