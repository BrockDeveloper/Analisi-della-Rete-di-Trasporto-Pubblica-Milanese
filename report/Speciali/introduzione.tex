
\chapter*{Introduzione}
\addcontentsline{toc}{chapter}{Introduzione}

La \textbf{Rete di Trasporto Pubblica Metropolitana Milanese} è una rete che ha avuto un'importante espansione negli ultimi decenni, arrivando ad avere 5 \textit{linee} diverse con molti punti di interscambio e un'estensione ramificata su tutto il territorio \textit{urbano} ed \textit{extra-urbano} \cite{ATM2025}.

Le reti di trasporto possono essere analizzate seguendo un approccio \textit{topologico}, in cui si può rappresentare come un \textbf{grafo} in cui i nodi e gli archi hanno una specifica controparte nella rete reale \cite{MattssonJenelius2015}.

L'obiettivo di questo studio è fornire una rappresentazione a \textit{grafo} della \textit{Rete Metropolitana Milanese}, per poi studiarne le \textbf{principali metriche} che la caratterizzano in un approccio dettato dalla \textbf{network analysis}.

Tali metriche non saranno fini a se stesse, ma verranno \textbf{analizzate} per comprendere come la rete è \textbf{strutturata}, quali sono i suoi \textbf{vantaggi} rispetto al \textit{trasporto pubblico} e quali, invece, sono le sue \textbf{debolezze} che possono essere attaccate.

Su queste debolezze verranno poi svolte delle \textbf{analisi di vulnerabilità}, con l'obiettivo di capire quanto la rete è \textbf{resistente} e \textbf{resilienze} a diverse tipologie di \textit{attacchi} e \textit{guasti}.

\section*{Specifiche del progetto}
\addcontentsline{toc}{chapter}{Specifiche del progetto}
Tutto le analisi e le elaborazioni presenti in questo elaborato hanno una corrispondenza in \textbf{codice}, in particolare ogni capitolo corrisponde ad un \textit{notebook Python} differente: 

\begin{center}
ogni \textit{notebook} ha nel proprio nome i numeri di capitolo ai quali si riferisce.
\end{center}

Oltre a quanto specificato nell'Introduzione, le \textbf{domande} poste in successione nello svolgimento di questo progetto e di questo elaborato sono:
\begin{itemize}
    \item Come rappresentare una rete di trasporto pubblico come un grafo, in termini di metodologie e corrispondenze tra nodi e archi e elementi reali della rete;
    \item Come analizzare il grafo sfruttando la \textit{network analysis}, decidendo quali metriche e quali analisi sono utili e propedeutiche alla domanda successiva;
    \item Come svolgere uno studio di \textit{vulnerabilità} della rete, per comprendere in situazioni reali quali \textit{attacchi} sono più o meno aggressivi nei confronti della rete.
\end{itemize}

\section*{Motivazioni}
\addcontentsline{toc}{chapter}{Motivazioni}
Avendo svolto diversi progetti di \textit{data analysis} nel corso della \textit{triennale} e della prima parte della \textit{magistrale}, ho preferito svolgere un progetto di \textit{network analysis} per affrontare nuovi tipi di competenze.